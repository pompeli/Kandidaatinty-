\section{Johdanto}

Aivojen sähköistä aktiivisuutta voidaan seurata noninvasiivisesti elektroenkefalografian (EEG) ja magnetoenkefalografian (MEG) avulla. EEG mittaa neuronien sähköistä toimintaa päänahan kautta mitattavien potentiaalierojen avulla. MEG puolestaan mittaa sähköisen toiminnan aikaansaamia magneettikenttiä pään ulkopuolelta. \citep{Hamalainen1993MagnetoencephalographytheoryBrain}

Lähteenpaikannuksen avulla yritetään saada selville aivoaktiivisuuden sijainti MEG:n tai EEG:n avulla saatujen mittausten avulla. Tällaista ulkoisesti saatujen mittausten avulla tehtävää lähteenpaikannusta sanotaan käänteisongelmaksi (\textit{inverse problem}) \citep[s. 2]{hansen2010meg}. Lääketieteellisessä tutkimuksessa on hyvin tärkeää tietää erilaisten ärsykkeiden aikaansaamia aktivaatioalueita aivokuorella. 

Usean signaalin luokittelu (\textit{multiple signal classification, MUSIC}) \citep{Schmidt1986MultipleEstimation} on aliavaruuksien käyttöön perustuva lähteenpaikannusalgoritmi. Algoritmissa mittausdata jaetaan keskenään ortogonaalisiin aliavaruuksiin, signaali- ja kohina-avaruuksiin \citep{Schmidt1986MultipleEstimation, Mosher1999SourceMUSIC}. Tämän jälkeen tarkistetaan tietyn lähdepisteen kuuluminen signaaliavaruuteen ortogonaaliprojektion avulla \citep{Mosher1999SourceMUSIC}. Laskennallinen tehokkuus ja epäherkkyys kohinalle tekevät siitä hyvän menetelmän lähteenpaikannukselle \citep{Makela2018TruncatedLocalization}. MUSIC-algoritmia voidaan käyttää MEG- tai EEG-mittausdatan analysoinnissa.

Tämän kandidaattityön tarkoituksena on esitellä MUSIC-algoritmi ja sen iteratiiviset versiot, RAP- sekä TRAP-MUSIC. Algoritmeja testataan simuloidulla datalla aika-alueella pallomallin avulla sekä oikean mittausdatan avulla oikeiden aivojen mallilla. Työssä esitellään myös MUSIC-algoritmin käyttöä taajuusalueella.

Luvussa 2 esitellään taustaa sähkömagneettisista ilmiöistä aivoissa. Luvussa 3 käydään läpi vaadittavia matemaattisia näkökulmia ja esitellään MUSIC-algoritmi sekä siihen perustuvia iteratiivisia versioita. Luvussa 4 esitellään MUSIC-algoritmin simulointia sen eri versioilla pallomallilla ja oikealla päämallilla.