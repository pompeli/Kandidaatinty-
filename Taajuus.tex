\subsection{MUSIC-algoritmi taajuusalueella}

MUSIC-algoritmilla voidaan löytää tietyllä taajudella olevia lähteitä. Tietyn taajuuden valinta annetusta jonosta tapahtuu diskreetin Fourier-muunnoksen (DFT) avulla. Taajuusanalyysissä mittausaika jaetaan peräkkäisiin epookkeihin, jotka ovat saman pituisia ja osittain päällekkäin.

Datamatriisin \textbf{Y} rivejä $\mathbf{Y}(j,:), j=1,...,m$ voidaan kuvata näytejonoina, joista on otettu tasavälisesti näytteet ajanhetkinä $t_1,...,t_M$ taajuudella $f_s$. Tällöin näytteiden aikaväli on $d = f_s^{-1}$.

Diskreettiä Fourier-muunnosta sovelletaan näytejonoon $x(n)$, jonka pituus on $2N+1$. Olkoon $X(m)$ diskreetti Fourier-muunnos tälle näytejonolle siten, että:

\begin{equation}
    X(m) = \sum_{n=0}^{2N}x(n)e^{-i\frac{2\pi}{2N+1}mn},
\end{equation}
jossa $i$ on imaginaariyksikkö ja $m$ on jokin kokonaisluku.

Näytejonon pituus valitaan pieneksi parittomaksi kokonaisluvuksi siten, että $(2N+1)d = T$, jossa $T$ on näytejonon ajan pituus.

Datajono jaetaan epookkeihin, joiden lukumäärä olkoon $N_E$ ja pituus $P$. Epookkien päällekkäisten indeksien lukumäärä olkoon $L$. Epookit jaetaan siten, että indeksillä $j$ oleva epookki sisältää näytepisteet $E_j = \{p_j,p_{j+1},...,q_j\}$, jossa $j=1,...,N_E$, $p_j = (j-1)(P-L)$ ja $q_j = p_j+P-1$. 

Olkoon $\mathbf{Y}(:,E_j)$ epookkia $E_j$ vastaava datamatriisin \textbf{Y} osamatriisi. Vastaavasti voidaan merkitä $\mathbf{S}(:,E_j)$ ja $\text{\boldmath$\varepsilon$}(:,E_j)$ kuvaamaan alkuperäisten matriisien osamatriiseja. Olkoon $F$ lineaarinen operaattori, joka poimii haluttua taajuutta $f$ vastavat termit jonosta $x(n)$. $F$ määritellään yhtälöllä:
\begin{equation}
    F(x) = X(fT)=\sum_{n=0}^{2N}x(n)e^{-i\frac{2\pi}{2N+1}fTn}.
\end{equation}

Olkoon lähteiden määrä $n$. Kun operaattoria $F$ käytetään datamatriisin osamatriisiin $\mathbf{Y}(:,E_j)$ yhteen riviin $k$, saadaan
\begin{equation}
    \begin{split}
    F(\mathbf{Y}(k,E_j))& = F(\sum_{h=1}^n\mathbf{A}(k,h)\mathbf{S}(h,E_j))+F(\text{\boldmath$\varepsilon$}(k,E_j))\\
    & = \sum_{h=1}^n\mathbf{A}(k,h)F(\mathbf{S}(h,E_j))+F(\text{\boldmath$\varepsilon$}(k,E_j))
    \end{split}
\end{equation}
Näin saadaan muodostettua frekvenssialueen yhtälö:
\begin{equation}
    \mathbf{Z = AR+\text{\boldmath$\tau$}},
\end{equation}
jossa $\mathbf{Z}\in \mathbb{C}^{m\times N_E}$ on taajuutta $f$ vastaava taajuusalueen datamatriisi, $\mathbf{R}\in \mathbb{C}^{n\times N_E}$ on frekvenssikulkumatriisi ja $\text{\boldmath$\tau$}\in \mathbb{C}^{m\times N_E}$ on taajuusalueen kohinamatriisi. $\mathbf{R}(h,k) = F(\mathbf{S}(h,E_j))$ kuvaa lähteen $h$ matriisista \textbf{S} saatua taajuutta $f$ vastaavaa termiä epookissa $E_j$.

Saatuun yhtälöön voidaan soveltaa kompleksista MUSIC-algoritmia, jolla löydetään taajuutta $f$ vastaavat lähteet. Tällöin frekvenssikulkumatriisin \textbf{R} taajuutta $f$ vastaavien rivien on oltava lineaarisesti riippumattomia.