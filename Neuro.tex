\section{Taustaa}

\subsection{Aivojen rakenne ja toiminta}
Aivot koostuvat neuroneista eli hermosoluista, joiden tärkein tehtävä on kuljettaa sähköistä informaatiota toisille soluille. Sooma on neuronin toiminnallinen keskus, josta signaali kulkeutuu eteenpäin viejähaaraketta eli aksonia pitkin. Dendriitit eli tuojahaarakkeet vastaanottavat ja antavat eteenpäin ärsykkeitä toisilta neuroneilta. \citep{Hamalainen1993MagnetoencephalographytheoryBrain}

Neuronin aksonia pitkin kulkeutuu sähköinen hermoimpulssi eli aktiopotentiaali. Aktiopotentiaali kulkeutuu amplitudiaan muuttamatta neuronin päädyssä oleviin synapseihin, joista signaali välittyy kemiallisesti toiselle neuronille välittäjäaineiden avulla. Kun välittäjäaineet saapuvat synapsin toiseen päätyyn, syntyy postsynaptinen potentiaali (PSP). \citep{Hamalainen1993MagnetoencephalographytheoryBrain}

Aivokuori eli korteksi on aivojen päällimmäinen, poimuttunut kerros. Pyramidisolut sijaitsevat korteksilla ja ne ovat kohtisuorassa korteksin pintaa vastaan. \citep{Hamalainen1993MagnetoencephalographytheoryBrain}

\subsection{MEG}
Sähkövirrat neuroneissa synnyttävät ympärilleen magneettikenttiä Maxwellin yhtälöiden mukaan. MEG mittaa pään ulkopuolelta neuronien synnyttämiä heikkoja magneettikenttiä, joilla on kohtisuora komponentti pään pintaa vastaan. Täten vain neuronit, jotka ovat samansuuntaisia pään pinnan kanssa tuottavat pään ulkopuolelle havaittavaa magneettikenttää. Täten MEG:llä tutkitaan lähteitä, jotka ovat korteksin poimuissa. \citep[s. 5]{hansen2010meg}

\subsection{Aivojen tutkiminen}

Yksittäisen neuronin aikaansaamaa magneettikenttää on liian heikko havaita MEG-laitteella. Tuhansien pyramidisolujen synkronoitu toiminta saa aikaan pään ulkopuolella havaittavan magneettikentän. Aivojen neurofysiologista aktiivisuutta voidaan kuvata sähköisinä virtadipoleina. 

Lähteen tuottaman magneetti- tai sähkökentän matemaattista laskemista pään ulkopuolella kutsutaan suoraksi ongelmaksi (\textit{forward problem}). Tietyn lähteen paikantaminen pään ulkopuolelta saatujen arvojen perusteella sanotaan käänteisongelmaksi (\textit{inverse problem}).

MUSIC-algoritmiin perustuvat lähdepaikannusmenetelmät vaativat suoran mallin (\textit{forward model}) toimiakseen. Suorassa mallissa täytyy ottaa huomioon pään ja aivojen geometriat sekä päänahan, kallon ja aivojen johtavuudet. \citep{hansen2010meg}

Neuromagneettista toimintaa voidaan mitata MEG:n avulla ainoastaan aivokuorelta, mikä johtaa siihen, ettei lähteen sijainnille saada yksikäsitteistä ratkaisua \citep[s. 2]{hansen2010meg}.
