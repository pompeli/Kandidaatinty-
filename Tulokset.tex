\section{Tulokset}
Tässä kandidaattityössä käytettiin Matlab-ohjelmaa simuloimaan MUSIC-algoritmia ja sen eri versioita. Simulaatioita tehtiin yksinkertaisella pallomallilla sekä todellisten aivojen mallilla. Apukoodit simulointiin on saatu Jukka Sarvakselta sekä Matti Stenroosilta.

Pallomallissa pään osat mallinnetaan kolmena johdepallona. Uloin pallo kuvaa päänahkaa, seuraava kalloa ja sisin pallo aivoja ja sen pinta aivokuorta. Jokaisella pallolla on homogeeninen johtavuus. 
Elektrodit sijoitetaan uloimman pallokuoren pinnalle ja lähdedipolit sisimmän pallokuoren pinnalle. Lähdepisteet voidaan projisoida tasolle yksikköympyrälle ja MUSIC-algoritmin avulla muodostetaan topografia paikannusfunktion arvoista. Aivojen ja päänahan johtavuuksiksi valittiin 0,33 ja kallon johtavuudeksi 0,0042.

Kuvissa \ref{fig:mfix} ja \ref{fig:mfree} on simuloitu MUSIC-algoritmia sekä kiinnitetyllä että vapaalla orientaatiolla. Simuloinneissa kolme lähdedipolia sijoitettiin aivokuorelle, jotka on merkitty punaisina renkaina ja topografia on muodostettu MUSIC-algoritmin avulla.

Kuvissa \ref{fig:RAPfix} ja \ref{fig:RAPfree} on simuloitu RAP-MUSIC- algoritmia kiinnitetyllä ja vapaalla orientaatiolla. Kuvissa musta rasti kuvaa paikannusfunktion maksimiarvon paikkaa ja löydetyt dipolit merkataan punaisella neliöllä.

Kuvista huomataan, että MUSIC vapaalla orientaatiolla saa aikaan suuria aktiivisuuden alueita, mikä vaikeuttaa lähteiden paikantamista.

Tavallisen MUSIC-algoritmin yksi ongelmista on oikeiden lähteiden erottaminen vääristä \citep{Mosher1998RecursiveLocalization}. Lisäksi MUSIC-algoritmilla on vaikeuksia löytää synkronoituja lähteitä \citep{Mosher1999SourceMUSIC}. RAP- sekä TRAP-MUSIC-algoritmeilla vaikeuksia löytää hyvin lähellä toisiaan olevia dipoleita. Löydettyä lähdettä projisoidessa pois häviää myös löytämättä jääneen lähteen komponentteja.

\clearpage
\begin{figure}[h]
    \centering
    \includegraphics[scale=0.38]{mfix.jpg}
    \caption{MUSIC-algoritmin simulointi pallomallilla kiinnitetyllä orientaatiolla.}
    \label{fig:mfix}
\end{figure}

\begin{figure}[h]
    \centering
    \includegraphics[scale=0.4]{mfree.jpg}
    \caption{MUSIC-algoritmin simulointi pallomallilla vapaalla orientaatiolla.}
    \label{fig:mfree}
\end{figure}

\clearpage

\begin{figure}[ht]
    \centering
    \includegraphics[width=\textwidth]{RAPfixed.jpg}
    \caption{RAP-MUSIC kiinnitetyllä orientaatiolla}
    \label{fig:RAPfix}
\end{figure}

\clearpage
\begin{figure}[ht]
    \centering
    \includegraphics[width=\textwidth]{RAPfree.jpg}
    \caption{RAP-MUSIC vapaalla orientaatiolla}
    \label{fig:RAPfree}
\end{figure}