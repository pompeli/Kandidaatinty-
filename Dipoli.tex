\subsection{Aivojen sähköinen toiminta}
Neuroni eli hermosolu vastaa aivojen sähköisen viestinnän kuljetuksesta hermoimpulssien välityksellä. Hermoimpulssia voidaan kuvata sähkövirtana, joka kulkeutuu neuronin sisällä. Tämä sähkövirta synnyttää neuronin ulkopuolelle sähkökentän joka taas synnyttää ympärilleen Maxwellin yhtälöiden mukaisesti magneettikentän.

Sähkö- ja magneettikenttien voidaan ajatella syntyvän päälähdevirtajakaumasta $\mathbf{J^p}$ \citep{Sarvas1987Basic}. Paikassa $\mathbf{r'}$ aktivoituneen alueen aiheuttamaa päälähdevirtajakaumaa paikassa \textbf{r} voidaan mallintaa virtadipolin \textbf{Q} avulla:
\begin{equation}
    \mathbf{J^p(r) = Q\delta (r-r')},
\end{equation}
jossa $\delta (\mathbf{r-r'})$ on Diracin deltafunktio \citep{Baillet2001ElectromagneticMapping}.


