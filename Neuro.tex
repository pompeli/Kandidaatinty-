\section{Teoreettinen tausta}

\subsection{Aivojen sähköinen toiminta}
Neuroni eli hermosolu vastaa aivojen sähköisen viestinnän kuljetuksesta hermoimpulssien välityksellä. Hermoimpulssia voidaan kuvata sähkövirtana, joka kulkeutuu neuronin sisällä. Tämä sähkövirta synnyttää neuronin ulkopuolelle sähkökentän joka taas synnyttää ympärilleen Maxwellin yhtälöiden mukaisesti magneettikentän. 

Aivojen korteksilla olevat pyramidisolut ovat kohtisuorasti suuntautuneita korteksin pintaa vastaan ja niiden synkronoitu aktivoituminen synnyttää pään ulkopuolella havaittavia signaaleja \citep{He2018ElectrophysiologicalDynamics}. Pyramidisolujen toimintaa voidaan mallintaa virtadipoleina \citep[s. 10]{HariMEGprimer}. 

Aktiivuus aivoissa voidaan ajatella syntyvän päävirrasta (\textit{primary current}) $\mathbf{J^p}$ \citep{Baillet2001ElectromagneticMapping}. Virtadipolia \textbf{Q} voidaan approksimoida pistemäisenä päävirtana \citep{Hamalainen1993MagnetoencephalographytheoryBrain}. Paikassa $\mathbf{r'}$ aktivoituneen alueen aiheuttamaa päävirtaa paikassa \textbf{r} voidaan mallintaa virtadipolin \textbf{Q} avulla:
\begin{equation}
    \mathbf{J^p(r) = Q\delta (r-r')},
\end{equation}
jossa $\delta (\mathbf{r-r'})$ on Diracin deltafunktio \citep{Baillet2001ElectromagneticMapping}. 



\subsection{Aivojen sähkömagneettisten signaalien mittaaminen}

MEG ja EEG ovat ei-invasiivisia aivokuvantamismenetelmiä, jotka perustuvat neuronien toiminnasta syntyvien sähkömagneettisten signaalien mittaamiseen aivojen ulkopuolelta. Ne mittaavat lähinnä pyramidisolujen postsynaptisia potentiaaleja, joita voidaan mallintaa virtadipoleina. Näiden menetelmien mittaamat signaalit kuvaavat suoraan aivojen sähköistä toimintaa \citep{Baillet2001ElectromagneticMapping}. EEG ja MEG kuvavat hyvällä aikaresoluutiolla mutta heikolla spatiaalisella resoluutiolla \citep{He2018ElectrophysiologicalDynamics}.

MEG:n avulla mitataan lähinnä aktivaatioita neuroneista, jotka sijaitsevat aivokuoren uurteiden seinillä, sillä nämä neuronit tuottavat pään ulkopuolella havaittavaa magneettikenttää. \citep[s. 4--5]{hansen2010meg} MEG on hyvin herkkä ulkopuolisille häiriöille, minkä takia MEG-laite yleensä sijoitetaan suojahuoneeseen, jonne ulkopuoliset magneettikentät eivät pääse sisälle. Myös, esimerkiksi, silmien räpäyttäminen ja sydämen lyönti aiheuttavat häiriötä mittauksessa. \citep{Hamalainen1993MagnetoencephalographytheoryBrain}

EEG mittaa aivojen aktiivisuutta päänahalle asetettujen elektrodien avulla. Nämä elektrodit mittaavat potentiaalieroja tiettyyn vertailuelektrodiin verrattuna. \citep{Michel2004EEGImaging} Toisin kuin MEG:n tapauksessa, EEG:n mittaamiin sähköisiin potentiaalieroihin pään pinnalla vaikuttavat myös pään pintaa kohtisuorat lähteet \citep[s. 10]{HariMEGprimer}.
