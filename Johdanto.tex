\section{Johdanto}
Aivojen toiminta perustuu sähkökemiallisiin vuorovaikutuksiin hermosolujen eli neuronien kautta. Kynnyksen ylittävä ärsyke saa aikaan hermoimpulssin, joka kulkee hermosolua pitkin ja signaali välittyy neuronilta toiselle synapsin kautta.\cite{} 
Elektroenkefalografian (EEG) avulla saadaan mitattua aivojen potentiaalierojen vaihtelua hyvällä aikaresoluutiolla. EEG:tä käytetään usein epilepsian toteamiseen sekä unen analyysiin. \cite{Nunez2006ElectricEEG}

Neuronien tuottamat sähkövirrat saavat aikaan ympärilleen heikkoja magneettikenttiä, jotka voidaan mitata pään ulkopuolelta magnetoenkefalografialla (MEG) \cite{Hamalainen1993MagnetoencephalographytheoryBrain}. MEG-laitteessa käytetään suprajohtavia SQUID-antureita. 

EEG:n ja MEG:n avulla saadaan tietoa aivojen aktiivisuudesta, mutta usean signaalilähteen yhtäaikainen toiminta aiheuttaa hankaluuksia paikantaa tietty lähde. Monenlaisia menetelmiä lähteen paikantamiselle on keksitty,joista yksi merkittävimmistä on MUSIC-algoritmi (Multiple Signal Classification), joka perustuu mittausdatan aliavaruuksien käyttöön. MUSIC-algoritmilla pystytään mittaamaan useita signaaleja yhtäaikaisesti suurella resoluutiolla. \cite{Mosher1999SourceMUSIC}

Tämän kandidaattityön tarkoituksena on selvittää, miten MUSIC-algoritmilla saadaan paikannettua signaalin lähteitä EEG- ja MEG-datan analyysissä. Mittausdataa kerätään sekä aika- että taajuusalueella ja analysoidaan MUSIC-algoritmilla MATLAB-ohjelman avulla. Työssä perehdytään myös MUSIC-algoritmin iteratiivisiin menetelmiin, RAP- sekä TRAP-MUSIC -algoritmeihin.
