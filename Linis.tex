\subsection{Lineaarialgebran käsitteitä ja tuloksia}

Olkoon $\mathbb{R}^m$ \textit{m}-uloitteinen vektoriavaruus. Vektorit $\{\mathbf{a}_1,...,\mathbf{a}_n\}$ virittävät $\mathbb{R}^m$:n aliavaruuden, jota merkitään $\text{span}(\mathbf{a}_1,...,\mathbf{a}_n)$. Jos vektorit $\{\mathbf{a}_1,...,\mathbf{a}_n\}$ ovat lineaarisesti riippumattomat, muodostavat ne aliavaruuden $\mathbb{V}$ kannan.

Vektorit $\{\mathbf{a}_1,...,\mathbf{a}_n\}$ ovat keskenään ortogonaalisia, jos ne ovat kohtisuorassa toisiaan vastaan. Ne ovat lisäksi ortonormaaleja, jos ne ovat yksikkövektoreita eli $||\mathbf{a}_i||=1$, jossa $||\cdot||$ kuvaa vektorin normia.

Matriisin $\mathbf{A}\in \mathbb{R}^{m\times n}$ rivien lukumäärä on $\mathit{m}$ ja sarakkeiden $\mathit{n}$. Merkintä $\mathbf{A}(:,i)$ tarkoittaa matriisin \textbf{A} \textit{i}:ttä saraketta ja vastaavasti $\mathbf{A}(i,:)$ \textit{i}:ttä riviä. Matriisin \textbf{A} lähdeavaruuden $\text{span}(\mathbf{A})$ virittävät sen sarakevektorit eli $\text{span}(\mathbf{A}) = \text{span}(\mathbf{A}(:,1),...,\mathbf{A}(:,n)) = \text{span}(\mathbf{A}(:,1:n))$. 

Matriisin \textbf{A} aste eli $\text{rank}(\mathbf{A})$ kuvaa matriisin lineaarisesti riippumattomien sarake- tai rivivektoreiden lukumäärää. Neliömatriisi $\mathbf{A}\in \mathbb{R}^{m\times m}$ on kääntyvä eli sillä on käänteismatriisi $\mathbf{A}^{-1}$, jos matriisin aste on yhtä kuin sen rivien tai sarakkeiden määrä eli $r = \text{rank}(\textbf{A})=m$.

Singulaariarvohajotelman avulla matriisi $\mathbf{A}\in \mathbb{R}^{m\times n}$ voidaan esittää diagonaalimatriisin \textbf{D} sekä ortogonaalisten matriisien \textbf{U} ja \textbf{V} avulla:

\begin{equation}
    \mathbf{A = UDV^T,}
    \label{eq:svd}
\end{equation}
jossa matriisi \textbf{D} sisältää matriisin \textbf{A} singulaariarvot. Matriisi \textbf{U} muodostuu matriisin \textbf{A} vasemmanpuoleisista singulaarivektoreista ja \textbf{V} vastaavasti oikeanpuoleisista singulaarivektoreista.

Singulaariarvot $d_1,...,d_p$, $p=\text{min}(m,n)$ muodostavat jonon $d_1 \geq d_2 \geq ... \geq d_r \geq d_{r+1} = ... = d_p = 0$, jossa $r = \text{rank}(\mathbf{A})$. Siten singulaariarvohajotelma voidaan esittää matriisitulon avulla: 

\begin{equation}
    \mathbf{A} = \mathbf{U}(:,1:r)\mathbf{D}(1:r,1:r)\mathbf{V}(:,1:r)^T
    \label{eq:2}
\end{equation}


Pseudokäänteismatriisi $\mathbf{A^{\dagger}}$ on yleistys matriisin käänteismatriisille. Tämä joudutaan muodostamaan silloin, kun matriisi $\mathbf{A}$ ei ole kääntyvä. Pseudokäänteismatriisi saadaan muodostettua singulaariarvohajotelman avulla:

\begin{equation}
    \mathbf{A^{\dagger} = VD^{\dagger}U^T},
\end{equation}
jossa \textbf{U} ja \textbf{V} ovat kaavasta \ref{eq:svd} ja $\mathbf{D}^{\dagger}$ on $r\times r$-diagonaalimatriisi, jonka diagonaaliarvot ovat $d^{-1}_1,...,d^{-1}_r$.

Ortogonaaliprojektio aliavaruuteen $\text{span}(\mathbf{A})$ saadaan muodossa: 

\begin{equation}
    \mathbf{P}_{\text{span}(\mathbf{A})}= \mathbf{AA}^{\dagger}
\end{equation}

Oletetaan, että vektorit $\{\mathbf{a}_1,...,\mathbf{a}_n\} \in \mathbb{R}^m$ muodostavat ortonormaalin kannan aliavaruudelle $\mathbb{V}$. Olkoon $\mathbf{A} = [\mathbf{a}_1,...,\mathbf{a}_n]$. Koska ortogonaaliselle matriisille \textbf{A} pätee $\mathbf{A}^{\dagger} = \mathbf{A}^T$, niin ortogonaaliprojektio aliavaruuteen $\mathbb{V}$:

\begin{equation}
    \mathbf{P}_{\mathbb{V}} = \mathbf{AA}^T
\end{equation}
ja tämän ortogonaaliseen aliavaruuteen $\mathbb{V}^{\bot}$ ortogonaaliprojektio:

\begin{equation}
    \mathbf{P}_{\mathbb{V}^{\bot}}=\mathbf{I}-\mathbf{AA}^T,
\end{equation}
jossa \textbf{I} on identiteettimatriisi.

Ortogonaalisen matriisin \textbf{A} sarakkeet ja rivit ovat ortonormaaleja, mikä johtuu siitä, että $\mathbf{U}\mathbf{U}^T = \mathbf{I}$, koska $\mathbf{U}^T = \mathbf{U}^{-1}$. Olkoon $\mathbf{A = UDV^T}$ ja $r = \text{rank}(\mathbf{A})$. Silloin ortogonaalisen matriisin \textbf{U} sarakkeet $\mathbf{U}(:,1),...,\mathbf{U}(:,r)$ muodostavat ortonormaalin kannan $\text{span}(\mathbf{A})$:lle eli $\text{span}(\mathbf{A}) = \mathbf{U}(:,1:r)$ \citep{Uusitalo1997Signal-spaceComponents}. Projektio aliavaruuteen $\text{span}(\mathbf{A})$ voidaan nyt muodostaa singulaariarvohajotelman avulla muodossa:

\begin{equation}
    \mathbf{P}_{\text{span}(\mathbf{A})} = \mathbf{U}(:,1:r)\mathbf{U}(:,1:r)^T.
    \label{eq:projektio}
\end{equation}
