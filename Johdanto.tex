\section{Johdanto}

Aivojen sähköistä aktiivisuutta voidaan seurata noninvasiivisesti elektroenkefalografian (EEG) ja magnetoenkefalografian (MEG) avulla. EEG mittaa neuronien sähköistä toimintaa päänahalla mitattavien potentiaalierojen avulla. MEG puolestaan mittaa sähköisen toiminnan aikaansaamia magneettikenttiä pään ulkopuolelta. \citep{Hamalainen1993MagnetoencephalographytheoryBrain}

Lähteenpaikannuksella pyritään selvittää aivoaktiivisuuden sijainti mittauksista saatujen signaalien perusteella. Tällaista ulkoisesti saaduista mittauksista tehtävää lähteenpaikannusta sanotaan käänteisongelmaksi (\textit{inverse problem}) \citep[s. 2]{hansen2010meg}. Tälle käänteisongelmalle ei löydetä yksikäsitteistä ratkaisua \citep{Sarvas1987Basic}. Lääketieteellisessä tutkimuksessa on hyvin tärkeää tietää, missä aivokuoren osassa aktivaatio tapahtuu. 
%tilavuusjohdemalli

Usean signaalin luokittelu (\textit{multiple signal classification, MUSIC}) \citep{Schmidt1986MultipleEstimation} on aliavaruuksien käyttöön perustuva lähteenpaikannusalgoritmi. Algoritmissa mittausdata jaetaan keskenään ortogonaalisiin signaali- ja kohina-avaruuksiin \citep{Schmidt1986MultipleEstimation, Mosher1999SourceMUSIC}. Tämän jälkeen tutkitaan tietyn lähdepisteen kuuluminen signaaliavaruuteen ortogonaaliprojektion avulla \citep{Mosher1999SourceMUSIC}. Laskennallinen tehokkuus ja epäherkkyys kohinalle tekevät siitä hyvän menetelmän lähteenpaikannukselle \citep{Makela2018TruncatedLocalization}. MUSIC-algoritmia voidaan käyttää MEG- tai EEG-mittausdatan analysoinnissa sekä aika- että taajuusalueella.

Tämän kandidaattityön tarkoituksena on esitellä MUSIC-algoritmi ja sen iteratiiviset versiot, RAP- sekä TRAP-MUSIC. Algoritmeja testataan simuloidulla datalla pallomallin avulla sekä oikean mittausdatan avulla. Työssä esitellään myös MUSIC-algoritmin käyttöä taajuusalueella.

Luvussa 2 esitellään taustaa sähkömagneettisista signaaleista. Luvussa 3 käydään läpi vaadittavia matemaattisia apuneuvoja ja esitellään MUSIC-algoritmi sekä siihen perustuvia iteratiivisia versioita. Luvussa 4 esitellään MUSIC-algoritmin simulointia sen eri versioilla pallomallilla ja oikealla päämallilla. Luvussa 5 käydään läpi saatuja tuloksia.