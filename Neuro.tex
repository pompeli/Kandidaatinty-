\section{Teoreettinen tausta}

\subsection{Aivojen rakenne ja toiminta}
Aivot ovat keskushermoston tärkein osa ja ne vastaavat tiedon prosessoinnista ja välityksestä. Aivot koostuvat kahdesta aivopuoliskosta, jotka jaetaan aivolohkoihin. Eri aivolohkot vastaavat tietyistä aistimuksista. Aivokuori eli korteksi on aivojen päällimmäisin, poimuttunut kerros. \citep{Hamalainen1993MagnetoencephalographytheoryBrain}

Sähköinen informaatio aivoissa kulkee hermosolujen eli neuronien välityksellä. Sooma on neuronin toiminnallinen keskus, josta signaali kulkeutuu eteenpäin viejähaaraketta eli aksonia pitkin. Dendriitit eli tuojahaarakkeet vastaanottavat ja antavat eteenpäin ärsykkeitä toisilta neuroneilta. \citep{Hamalainen1993MagnetoencephalographytheoryBrain}

Neuronin aksonia pitkin kulkeutuu sähköinen hermoimpulssi eli aktiopotentiaali. Aktiopotentiaali kulkeutuu amplitudiaan muuttamatta neuronin päätyyn, jossa kaksi neuronia ovat liittyneinä toisiinsa synapsilla. Aktiopotentiaalin saavuttua presynaptisen neuronin päätyyn vapautuu synapsirakoon välittäjäaineita, jotka välittävät viestiä eteenpäin postsynaptiselle neuronille. Välittäjäaineiden kiinnittyessä postsynaptisen neuronin päähän potentiaaliero neuronin kalvojännitteessä muuttuu. Tätä potentiaalimuutosta kutsutaan postsynaptiseksi potentiaaliksi (PSP), jota havaitaan MEG:n ja EEG:n avulla. \citep{Hamalainen1993MagnetoencephalographytheoryBrain}

\subsection{EEG}
EEG mittaa aivojen aktiivisuutta päänahalle asetettujen elektrodien avulla. Nämä elektrodit mittaavat potentiaalieroja tiettyyn referenssielektrodiin verrattuna. \citep{niedermeyer2011niedermeyer}

\subsection{MEG}
Sähkövirrat neuroneissa synnyttävät ympärilleen magneettikenttiä Maxwellin yhtälöiden mukaan. MEG mittaa pään ulkopuolelta näitä heikkoja magneettikenttiä suprajohtavien kvanttilaitteiden (SQUID) avulla.

MEG on hyvin herkkä ulkopuolisille häiriöille, minkä takia MEG-laite yleensä sijoitetaan suojahuoneeseen, jonne ulkopuoliset magneettikentät eivät pääse sisälle. Myös, esimerkiksi, silmien räpäyttäminen ja sydämen lyönti aiheuttavat häiriötä mittauksessa. \citep{Hamalainen1993MagnetoencephalographytheoryBrain}

\subsection{Aivojen tutkiminen}

Yksittäisen neuronin  aikaansaamaa magneettikenttää on liian heikko havaita MEG-laitteella. Tuhansien pyramidisolujen synkronoitu toiminta saa aikaan pään ulkopuolella havaittavan magneettikentän. \citep{He2018ElectrophysiologicalDynamics}

Aivojen sähköisiä aktiisuuden lähteitä voidaan approksimoida virtadipoleina, joissa varaukset kulkeutuvat lähteestä nieluun \citep[s. 6]{HariMEGprimer}. PSP:n aikaansaamaa sähkövirtaa voidaan pitää lyhyenä virtadipolina. \citep{He2018ElectrophysiologicalDynamics}

Neuromagneettista toimintaa voidaan mitata MEG:n avulla ainoastaan aivokuorelta, mikä johtaa siihen, ettei lähteen sijainnille saada yksikäsitteistä ratkaisua. Tämän vuoksi joudutaan rajoittamaan mahdollisia lähderatkaisuja \citep[s. 2]{hansen2010meg}.

