%%
%% MINKÄ PDF/A -standardin mukaan teen opinnäytetyöni?
%%
%% Ensisijaisesti PDF/A-1b -standardin mukaan. Kuvaajat ja kuvat mitä
%% tyypillisesti käytetään opinnäytetyössä eivät tarvitse
%% läpinäkyvyysominaisuuksia. Perus '2D' -näkymä on riittävä. Opinnäytetyössä
%% käytettävät fontit on määritelty tässä pohjassa eikä niitä pidä muuttaa. Jos
%% käytät kuvia, jossa läpikäkyvyysominaisuuksillä on merkitystä, käytä PDF/A-2b
%% -standardia. Älä käytä PDF/A-3b -standardia opinnäytetyössäsi.
%%
%%
%% MITKÄ kuvaformaatteja voin käyttää PDF/A-tiedoston tekemisessä?
%%
%% Kun käytät pdflatexia työsi kääntämisessä, käytä jpg-, png- tai pdf-formmaatia.
%% Pdf-muotoisten kuvien kanssa voi tulla ongelmia PDF/A -yhteesopivuuden kanssa.
%% Älä käytä PDF/A-formaattia kuvatiedostoissa.
%% Jos kuitenkin käytät latexia työsi kääntämisessä, ainoa sallittu kuvaformaatti
%% on eps. ÄLÄ käytä ps-formaattia kuvissasi.

%% KÄYTÄ näistä yhtä:
%% * ensimmäinen, jos käytät pdflatexia, joka kääntää tekstin suoraan
%%   pdf/a-tiedostoksi ja haluat julkaista opinnäytetyösi verkossa
%% * toinen, jos haluat tulostaa opinnäytteesi kansitettavaksi
%% * kolmas jos haluat tuottaa ps-tiedostoa ja siitä pdf/a:ta
%%
\documentclass[finnish, 12pt, a4paper, elec, utf8, a-1b, online]{aaltothesis}
%\documentclass[finnish, 12pt, a4paper, elec, utf8, a-1b]{aaltothesis}
%\documentclass[finnish, 12pt, a4paper, elec, dvips, online]{aaltothesis}
%% Kirjoita y.o. \documentclass optioiksi
%% korkeakoulusi näistä: arts, biz, chem, elec, eng, sci
%% editorisi käyttämä merkkikoodaustapa: utf8, latin1
%% opinnäytetyön kieli: finnish, english, swedish
%% tee arkistointikelpoista PDF/A-1b, PDF/A-2b tai PDF/A-3b pdf-tiedosto: a-1b,
%%                         a-2b, a-3b
%%                         (tavallinen pdf-tiedosto syntyy ilman a-*b optiota)
%% verkkoon menevä symmetrinen taitto ja sinisellä hypertekstillä: online  
%%                         (oletusarvo on leveä marginaali sivun sidonta puolella
%%                         ja musta hyperteksti)
%% kaksipuolinen tulostus: twoside (oletusarvo on yksipuolinen tulostus)
%%

%% Käytä yhtä näistä, jos kirjoitat englanniksi. Katso englanninokset
%% tiedostosta thesistemplate.tex.
%\documentclass[english, 12pt, a4paper, elec, utf8, a-1b, online]{aaltothesis}
%\documentclass[english, 12pt, a4paper, elec, utf8, a-1b]{aaltothesis}
%\documentclass[english, 12pt, a4paper, elec, dvips, online]{aaltothesis}

\usepackage{graphicx}
\usepackage{epstopdf}
\usepackage{natbib}

%% Matematiikan fontteja, symboleja ja muotoiluja lisää, näitä tarvitaan usein 
\usepackage{amsfonts,amssymb,amsbsy}

%% Korjaa vastaamaan korkeakouluasi, jos automaattisesti asetettu nimi on 
%% virheellinen 
%%
%% Change the school field to specify your school if the automatically 
%% set name is wrong
\university{aalto-yliopisto}
\school{Elämän korkeakoulu}

%% VAIN KANDITYÖLLE: Korjaa seuraava vastaamaan koulutusohjelmaasi
%%
\degreeprogram{Sähkötekniikan kandidaattiohjelma}
%%

%% Pääaineesi (kandi) tai professuuri (DI/MSc)
\major{Bioinformaatioteknologia}

%% Pääainekoodi (kandityö) tai professuurikoodi (DI/MSc- tai lisensiaatintyölle)
%%
\code{ELEC3016}
%%

%% 
%% Valitse yksi näistä kolmesta
%%
\univdegree{BSc}
%\univdegree{MSc}
%\univdegree{Lic}
%%

%% Oma nimi
%%
\thesisauthor{Teemu Teekkari}
%%

%% Opinnäytteen otsikko tulee tähän ja mahdollisesti uudelleen englannin- tai
%% ruostinkielisen abstraktin yhteydessä. Älä tavuta otsikkoa ja vältä liian
%% pitkää otsikkotekstiä. Jos latex ryhmittelee otsikon huonosti, voit joutua
%% pakottamaan rivinvaihdon \\ kontrollimerkillä.
%% Tällöin...
%% Muista että otsikkoja ei tavuteta! 
%% Jos otsikossa on ja-sana, se ei jää rivin viimeiseksi sanaksi vaan aloittaa
%% uuden rivin.
%% Anna ostikko uudelleen ilman rivinvaihtomerkkiä optionaalisena argumenttina
%% hakasuluissa. Näin tehdään, koska otsikko on osaa pdf/a-tiedostossa olevaa
%% metadataa, ja metadatassa ei saa olla rivinvaihtomerkkiä.
%%
\thesistitle{Opinnäyte}
%\thesistitle[Opinnäytteen otsikko]{Opinnäyteen\\ otsikko}
%%

%%
\place{Espoo}
%%

%% Kandidaatintyön päivämäärä on sen esityspäivämäärä! 
%% 
\date{12.7.2016}
%%

%% Kandidaattiseminaarin vastuuopettaja tai diplomityön valvoja.
%% Huomaa tittelissä "\" -merkki pisteen jälkeen, ennen välilyöntiä ja
%% seuraavaa merkkijonoa. 
%% Näin tehdään, koska kyseessä ei ole lauseen loppu, jonka jälkeen tulee 
%% hieman pidempi väli vaan halutaan tavallinen väli.
%%
\supervisor{TkT Matti Meikäläinen}
%%

%% Kandidaatintyön ohjaaja(t) tai diplomityön ohjaaja(t). Ohjaajia saa
%% olla korkeintaan kaksi.
%% 
\advisor{Prof.\ Tuomo Tietäväinen}
%\advisor{TkT Olli Ohjaaja}
%\advisor{DI Tina Tutkija}
%%

%% Aaltologo: syntaksi:
%% \uselogo{aaltoRed|aaltoBlue|aaltoYellow|aaltoGray|aaltoGrayScale}{?|!|''}
%% Logon kieli on sama kuin dokumentin kieli
%%
\uselogo{aaltoRed}{!}
%%

%% Suomenkielinen tiivistelmä:
%% Kaikki tiivistelmässä tarvittava tieto (nimesi, työnnimi, jne.) käytetään
%% niin kuin se on yllä määritelty.
%% Tiivistelmän avainsanat:
%% Huom! Avainsanat erotetaan toisistaan \spc -makrolla
%%
\keywords{}
%%

%% Tiivistelmän tekstiosa. Tämä teksti sisältyy pdf-tiedoston metadataa ja tulee
%% myös tiivistelmälomakkeeseen.
%% Kopioidaan tähän tiivistelmä tiedoston sisältö
\thesisabstract{
Tähän tulee kirjoittaa tiivistelmä, että se jää osaksi PDF:n metadataa. Tämän sisältö siirtyy automaattisesti tiivistelmän osioon.
}




%%

%% Tekijänoikeusteksti. Tekijänoikeus on tekijällä riippumatta siitä onko
%% copyright -merkintä näkyvissä vai ei. Halutessasi voit jakaa työsi Creative 
%% Commons -lisensillä (katso creativecommons.org), jolloin lisenssitekstin on
%% oltava näkyvissä. Kirjoita tähän haluamasi tekijänoikeustektin. Se kirjautuu
%% myös pdf-tiedoston metadataan.
%% Syntaksi:
%% \copyrigthtext{metadatateksti}{sivulle näkyviin tuleva teksti}
%%
%% A.o. metadataan menevässä tekstissä on käytettävä \noexpand -makroa ennen
%% \copyright -erikoismerkkiä ja lisäksi makrot (tässä \copyright ja \year) on
%% erotettava seuraavasta tekstistä \ -merkillä (välilyöntimerkki).
%% \copyrighttext-makron argumentissa olevat makrot automaattisesti hakevat
%% vuosiluvun ja tekijän nimi.
%% (Huom! \ThesisAuthor on aaltothesis.cls -tyylitiedoston sisäinen makro).
%% Toki saman tekstin olisi voinut kirjoittaa yksinkertaisesti näin:
%% \copyrighttext{Copyright \noexpand\copyright\ 2018 Teemu Teekkari}
%% {Copyright \copyright{} 2018 Teemu Teekkari}
%%
\copyrighttext{Copyright \noexpand\copyright\ \number\year\ \ThesisAuthor}
{Copyright \copyright{} \number\year{} \ThesisAuthor}
%%

%% Voit estää LaTeXia kirjoittamasta xmpdata-tiedostoon (sisältää pdf-tiedostoon
%% kirjoitettavaa metadataa) asettamalla writexmpdatafile lipun arvoksi 'false'.
%% Tämä mahdollistaa sen, että voit kirjoittaa metadataa suoraan oikeassa
%% muodossa tiedostoon opinnaytepohja.xmpdata.
%%
%\setboolean{writexmpdatafile}{false}
%%



%% Kaikki mikä paperille tulostuu, on tämän jälkeen
\begin{document}
	
%% Tehdään kansilehti
%%
\makecoverpage{}

%% Tehdään tekijänoikeusteksti näkyväksi.
%% Halutessasi voit jättää tekijänoikeustekstin pois luettavasta pdf-tiedostosta. 
%% Tämä voi tuntua hyvältä ajatukselta paperille tulostetulla versiossa eteenkin,
%% jos sivun ainoa teksti on "Copyright (c) vvvv Teemu Teekkari". Suositus on
%% kuitenkin jättää teksti näkyviin.
%%
%% \makecopyrightpage{}

%% Suomenkielinen tiivistelmä
%% Kaikki tiivistelmässä tarvittava tieto (nimesi, työnnimi, jne.) käytetään
%% niin kuin se on yllä määritelty.
%%


%% Tiivistelmän tekstiosa

%% \thesisabstract -makrossa kirjoitettu teksti on tallennettu \abstractext 
%% -makrosa jolloin voit siirtää metadataan menevä teksti sellaisenaan näin:
%%
\begin{abstractpage}[finnish]
	\abstracttext{}
\end{abstractpage}

\newpage

%% Sisällysluettelo
\thesistableofcontents

%% Opinnäytteessä jokainen osa alkaa uudelta sivulta, joten \clearpage

\clearpage

%% Johdanto haetaan erillisestä tiedostosta

\section{Johdanto (tulee vielä muuttumaan)}
Aivojen toiminta perustuu sähkökemiallisiin vuorovaikutuksiin hermosolujen eli neuronien kautta. Kynnyksen ylittävä ärsyke saa aikaan hermoimpulssin, joka kulkee hermosolua pitkin ja signaali välittyy neuronilta toiselle synapsin kautta.\cite{} 
Elektroenkefalografian (EEG) avulla saadaan mitattua aivojen potentiaalierojen vaihtelua hyvällä aikaresoluutiolla. EEG:tä käytetään usein epilepsian toteamiseen sekä unen analyysiin. \cite{Nunez2006ElectricEEG}

Neuronien tuottamat sähkövirrat saavat aikaan ympärilleen heikkoja magneettikenttiä, jotka voidaan mitata pään ulkopuolelta magnetoenkefalografialla (MEG) \cite{Hamalainen1993MagnetoencephalographytheoryBrain}. MEG-laitteessa käytetään suprajohtavia SQUID-antureita. 

EEG:n ja MEG:n avulla saadaan tietoa aivojen aktiivisuudesta, mutta usean signaalilähteen yhtäaikainen toiminta aiheuttaa hankaluuksia paikantaa tietty lähde. Monenlaisia menetelmiä lähteen paikantamiselle on keksitty,joista yksi merkittävimmistä on MUSIC-algoritmi (Multiple Signal Classification), joka perustuu mittausdatan aliavaruuksien käyttöön. MUSIC-algoritmilla pystytään mittaamaan useita signaaleja yhtäaikaisesti suurella resoluutiolla. \cite{Mosher1999SourceMUSIC}

Tämän kandidaattityön tarkoituksena on selvittää, miten MUSIC-algoritmilla saadaan paikannettua signaalin lähteitä EEG- ja MEG-datan analyysissä. Mittausdataa kerätään sekä aika- että taajuusalueella ja analysoidaan MUSIC-algoritmilla MATLAB-ohjelman avulla. Työssä perehdytään myös MUSIC-algoritmin iteratiivisiin menetelmiin, RAP- sekä TRAP-MUSIC -algoritmeihin.


%% Opinnäytteessä jokainen osa alkaa uudelta sivulta, joten \clearpage

\clearpage

%% Seuraava osio haetaan erillisestä tiedostosta

\input{AikaisempiTutkimus.tex}

%% Opinnäytteessä jokainen osa alkaa uudelta sivulta, joten \clearpage

\clearpage

%% Seuraava osio haetaan erillisestä tiedostosta

\section{Finding and referring to sources}

Never ever copy anything into your theses from somebody else's text
(nor your own previously published text). Never. Not even for starting
point to be rewritten later. The risk is that you forgot the copied
text to your thesis and end up to be accused of plagiarism. Plagiarism
is a serious crime in studies and science and can ruin your career
even its beginning. To repeate: never cut and paste text into your
thesis!

\subsection{Finding sources}

All work is based on someone else's work. You should find the relevant
sources of your field and choose the best of them. Also, you should
refer to the original source where a fact has been mentioned first
time. Remember source evaluation (criticism) with all sources you
find.

Good starting points for finding references in computer science are: 
\begin{itemize}
% You can use this command to set the items in the list closer to each other
% (ITEM SEParation, the vertical space between the list items) 
\setlength{\itemsep}{0pt}
\item Nelli Portal (Aalto Library): \url{http://www.nelliportaali.fi}
\item ACM Digital library: \url{http://portal.acm.org/}
\item IEEExplore: \url{http://ieeexplore.ieee.org}
\item ScienceDirect: \url{http://www.sciencedirect.com/}
\item \ldots although Google Scholar (\url{http://scholar.google.com/}) will
find links to most of the articles from the abovementioned sources, if you
search from within the university network
\end{itemize}

Some of the publishers do not offer all the text of the articles
freely, but the library has agreed on the rights to use the whole
text. Thus, you should sometimes use computers in the domain of the
university in order to get the full text. Sometimes the Nelli Portal
can also help getting the whole article instead of just the abstract.
The library has also brief instrucions how to find
information~\cite{howfindinfo}.

Instead of normal Google, use Google Scholar
(\url{http://scholar.google.fi/}). It finds academic publications whereas
normal Google find too much commercial advertisements or otherwise
biased information. Wikipedia articles should be referred to in the master
thesis only very, very seldomly. You can use Wikipedia for understanding
some basics and finding more sources, but often you cannot be sure if
the article is correct and unbiased.

One important part of the sources that you have found is the reference
list. This way you can find the original sources that all the other
research of the field refer. Often you can also find more information
with the name of the researchers that are often referred in the
articles.

\subsection{Referring to sources}

The main point in referring to sources is to separate your own
thinking and text from that of others. Facts of the research area can
be given without reference, but otherwise you should refer to
sources. This means two things: marking the source in the text where
it has been used, and listing the sources usually in the end of the
thesis in a way that help the reader to find the original source.

There are several bibliography styles, meaning how to form the
bibliography in the end of the thesis. Aalto's library has good
instructions for many styles~\cite{bibinstructions}. You should ask
from your supervisor or instructors which style you should use. This
thesis template uses the number style that is often used in software
engineering. The other style also used in the CS field,
e.g. usability, is the Harvard style where instead of numbers, the
reference is marked into the text with author's name and publishing
year. Other areas use also many other styles for making the lists and
marking the references.

In addition to the list in the end of the thesis, you have to mark the
source in the text where the source is used. There are three places
for the reference: in a sentence before the period, in the end of a
sentence after the period, or in the end of a paragraph. All of them
have different meaning. The main point is that first you paraphrase
the source using your own words and then mark the source. Next, we
give short examples that are marked with \emph{emphasised text}.

\emph{Haapasalo~\cite{HaapasaloThesis} researched database algorithms
  that allows use of previous versions of the content stored in the
  database.} This kind of marking means that this paragraph (or until
the next reference is given) is based on the source mentioned in the
beginning.  Giving the source you should use only the family name of
the first author of the article, and not give any hints about what is
the type of the article that is referred.

\emph{B+-trees offers one way to index data that is stored in to a
  database. Multiversion B+-trees (MVBT) offer also a way to restore
  the data from previous versions of the database. Concurrent MVBT
  allows many simultaneous updates to the database that is was not
  possible with MVBT.~\cite{HaapasaloThesis}} When the marking is
after the period, the reference is retrospective: all the paragraph
(or after previous reference marking) is based on the source given in
its end. If the content is very broad, you can start with saying
\emph{According to Haapasalo}, then continue referring the source with
several separate sentences, and in the end put the marking of your
source \emph{ that shows that CMVBT are the
  best. ~\cite{HaapasaloThesis}}. 

If your paragraph has several sources, the above mentioned styles are
not proper. The reader of your thesis cannot know which of your
sources give which of the statements. In this case, it is better to
use more finegraded refering where the reference markings that are
embedded in the sentences. For example, \emph{the multiversion B+-tree
  (MVBT) index of Becker et al.~\cite{becker:1996:mvbt} allows database
  users to query old versions of the database, but the index is not
  transactional.
  It's successor, the transactional MBVT (TMVBT), allows a single transaction
  running in its own thread or process to update the database concurrently
  with other transactions that only read the
  database~\cite{haapasalo:2009:tmvbt}. 
  Further development, titled the concurrent MBVT (CMVBT),
  allows several transactions to perform updates to the database at the same
  time~\cite{HaapasaloThesis}}. 
  Here, the references are marked before
  the period in the sentences where they are used.

Finally, direct quotes are allowed. However, often you should avoid
them since they do not usually fit in to your text very well. Using
direct quotes has two tricks: quotation marks and the source.  \emph{
  ``Even though deletions in a multiversion index must not physically
  delete the history of the data items, queries and range scans can
  become more efficient, if the leaf pages of the index structure are
  merged to retain optimality.''~\cite{HaapasaloThesis}} Quotes are
hard to make neatly since you should use only as much as needed
without changing the text. Moreover, you often do not really
understand what the author has mentioned with his wordings if you
cannot write the same with your own words. Remember also that never
cut and paste anything without marking the quotation marks right away,
and in general, never cut and paste anything at all!

Sometimes getting the original source can be almost impossible. In an
extremely desperate situation, you can refer with structure \emph{mr
  X~[\ldots] according to ms Y~[\ldots] defined that}, if you find a
source that refers to the original source. Note also that the
reference marking is never used as sentence element (example of how 
\textbf{not} to do it: \emph{\cite{HaapasaloThesis} describes
an optimal algorithm for indexing multiversiond databases.}).



%% Opinnäytteessä jokainen osa alkaa uudelta sivulta, joten \clearpage

\clearpage


%% Osan hienojaottelua alaosiin, eikä välttämättä edes tarpeen, tässä vain
%% esimerkkinä. Käytä harkintasi mukaan osan jaottelua, joskus alaotsikot
%% selventävät asioita ja joskus vain sirpaloittavat tarpeettomasti tekstiä.
%%  Jaottelu menee seuraavasti:
%% \section{osan otsikko} 
%% \subsection{alaotsikko}
%% \subsubsection{ala-alaotsikko}
%% Tätä pitemälle ei pidä jaotella. 

%% Lähteet

\bibliographystyle{agsm}    % Käytettävä viittausjärjestelmä, agsm = harvard
\bibliography{Lahteet}      % BibTeX-tietokanta, lahteet tiedoston nimi


%% Liitteet
%% Sivulaskurin viilausta opinnäytteen vaatimusten mukaan. Näkyy lähinnä
%% tiivistelmä lomakkeen sivunumerokentässä:
%% tekstin sivujen määrä + liiteiden sivujen määrä.
%% Poista a.o. \clearpage-, \storeinipagenumber- ja \thesisappendix -makrot, jos
%% liiteitä ei ole.
\clearpage
\storeinipagenumber

\thesisappendix

\section{Esimerkki liitteestä\label{LiiteA}}

Liitteet eivät ole opinnäytteen kannalta välttämättömiä ja 
opinnäytteen tekijän on 
kirjoittamaan ryhtyessään hyvä ajatella pärjäävänsä ilman liitteitä.
Kokemattomat kirjoittajat, jotka ovat huolissaan
tekstiosan pituudesta, paisuttavat turhan 
helposti liitteitä pitääkseen tekstiosan pituuden annetuissa rajoissa.
Tällä tavalla ei synny hyvää opinnäytettä.   

Liite on itsenäinen kokonaisuus, vaikka se täydentääkin tekstiosaa.
Liite ei siten ole pelkkä listaus, kuva tai taulukko, vaan 
liitteessä selitetään aina sisällön laatu ja tarkoitus. 

Liitteeseen voi laittaa esimerkiksi listauksia. Alla on 
listausesimerkki tämän liitteen luomisesta. 

%% Verbatim-ympäristö ei muotoile tai tavuta tekstiä. Fontti on monospace.
%% Verbatim-ympäristön sisällä annettuja komentoja ei LaTeX käsittele. 
%% Vasta \end{verbatim}-komennon jälkeen jatketaan käsittelyä.
\begin{verbatim}
	\clearpage
	\appendix
	\addcontentsline{toc}{section}{Liite A}
	\section*{Liite A}
	...
	\thispagestyle{empty}
	...
	tekstiä
	...
	\clearpage
\end{verbatim}

\end{document}
