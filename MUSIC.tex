\section{MUSIC-algoritmi}
\cite{Schmidt1986MultipleEstimation} kehitti lähteen paikannusmenetelmän, mikä tunnetaan nimellä Multiple Signal Classification (MUSIC). MUSIC-algoritmi perustuu mittausdatan jakamiseen keskenään ortogonaalisiin signaali- ja kohina-aliavaruuksiin, jonka jälkeen tarkistetaan potentiaalisen lähteen topografian kuuluminen signaaliavaruuteen \cite{Mosher1999SourceMUSIC}.

Olkoon mittauksista saatu data kerättynä matriisiin $\mathbf{Y}\in \mathbb{R}^{m\times N}$, jossa $\mathit{m}$ on mittausanturien määrä ja $\mathit{N}$ mittausten lukumäärä. Matriisilla $\mathbf{Y}$ on muoto

\begin{equation}
    \mathbf{Y=AS+\eta},
\end{equation}
jossa $\mathbf{A}$ on sekoitusmatriisi, $\mathbf{S}$ ajankulkumatriisi ja $\mathbf{\eta}$ mittauskohinaa. 

\begin{equation}
    \mu(\mathbf{p}) = \frac{||\mathbf{P}_s\mathbf{l(p)}||^2}{||\mathbf{l(p)}||^2}    
\end{equation}


\subsection{RAP-MUSIC}

\subsubsection{RAP-dilemma}
RAP-MUSICin paikannusfunktio saattaa löytää lähdepisteitä jo löydettyjen lähteiden läheltä, mikä johtuu siitä, ettei algoritmi pysty poistamaan topografiaa löydettyjen dipolien lähettyviltä \cite{Makela2018TruncatedLocalization}. Tätä RAP-dilemmaa havaitaan varsinkin kohinattomalla ja valkoisen kohinan datalla. Tämän algoritmin virheen korjaamiseksi \cite{Makela2018TruncatedLocalization} kehittivät ratkaisun, joka nimettiin TRAP-MUSICiksi. 

\subsection{TRAP-MUSIC}
Truncated RAP-MUSIC (TRAP-MUSIC) on muuten samankaltainen rekursiivinen algoritmi kuin RAP-MUSIC, mutta TRAP-MUSICin paikannusfunktioon on tehty pieni muutos: kun uusi lähde on löydetty, tämä kohta projisoidaan pois signaaliavaruudesta ja jäljellä oleva signaaliavaruus typistetään vastaamaan jäljellä olevien lähteiden arvioitua määrää \cite{Makela2018TruncatedLocalization}.