\subsection{MUSIC-algoritmi}
\cite{Schmidt1986MultipleEstimation} kehitti MUSIC-algoritmin, joka perustuu mittausdatan jakamiseen keskenään ortogonaalisiin signaali- ja kohina-aliavaruuksiin. Tämän jälkeen tarkistetaan potentiaalisen lähdepisteen topografian kuuluminen signaaliavaruuteen \citep{Mosher1999SourceMUSIC}. Tällä menetelmällä lähteet kuvataan virtadipoleina ja algoritmi vaatii päämalliin tehdyn suoran mallin (\textit{forward model}). Suorassa mallissa täytyy ottaa huomioon pään ja aivojen geometriat sekä päänahan, kallon ja aivojen johtavuudet \citep[s. 87]{hansen2010meg}.

Tässä työssä oletetaan dipolien olevan kiinnitettyjä eli niiden sijainnit eivät muutu ajan suhteen. Dipolit voivat olla joko kiinteästi orientoituneita eli niiden suunnat tiedetään etukäteen tai vapaasti orientoituneita, jolloin niiden suuntia ei tiedetä. MUSIC-algoritmit voidaan jakaa kiinteän tai vapaan orientaation algoritmeihin. Kiinteän orientaation MUSIC-algoritmia sanotaan myös skalaari-MUSIC:ksi (\textit{scalar MUSIC}) sekä vapaan orientaation vektori-MUSIC:ksi (\textit{vector MUSIC}). \citep{Makela2018TruncatedLocalization}

Olkoon pisteessä \textbf{p} dipoli, jolla on momentti $\mathbf{q} = s\text{\boldmath$\eta$}$, jossa $s$ on amplitudi ja \text{\boldmath$\eta$} on yksikkösuuntavektori. Tätä dipolia voidaan merkitä \textbf{(p,q)}. Mittaussensorien lukemien muodostamaa pystyvektoria merkitään $\mathbf{l(p,q)}$, jota sanotaan dipolin mittaussensorivasteeksi. Koska $\mathbf{l(p,q)}$ on lineaarinen amplitudin $s$ suhteen, saadaan $\mathbf{l(p,q)} = \mathbf{l(p,s\text{\boldmath$\eta$})} = s\mathbf{l(p,\text{\boldmath$\eta$})}$. Jos $s=1$, sanotaan $\mathbf{l(p,\text{\boldmath$\eta$})}$ dipolin topografiaksi.

Muodostetaan johtokenttämatriisi (\textit{lead field matrix}) pisteeseen \textbf{p}:

\begin{equation}
    \mathbf{L(p) = [l(p,e_x),l(p,e_y),l(p,e_z)]},
\end{equation}
jossa $\mathbf{e_x}$, $\mathbf{e_y}$ ja $\mathbf{e_z}$ ovat karteesisen koordinaatiston yksikkövektorit. Tällöin topografia saa muodon $\mathbf{l(p},\text{\boldmath$\eta$})=\mathbf{L(p})\mathbf{\text{\boldmath$\eta$}}$.

Olkoon mittauksista saatu data koottu matriisin $\mathbf{Y}\in \mathbb{R}^{m\times M}$, jossa \textit{m} on mittaussensorien määrä ja \textit{N} mittausten lukumäärä. Matriisin \textbf{Y} pylväs $\mathbf{Y}(:,j)$ on mittaussensorin $j$ mittaama data. Oletetaan dipolien lukumäärän olevan $n$. Tällöin matriisi $\mathbf{Y}$ voidaan kirjoittaa muodossa:

\begin{equation}
    \mathbf{Y=AS+\text{\boldmath$\varepsilon$}},
\end{equation}
jossa $\mathbf{A}\in \mathbb{R}^{m\times n}$ on sekoitusmatriisi (\textit{mixing matrix}), $\mathbf{S}\in \mathbb{R}^{n\times M}$ ajankulkumatriisi (\textit{time-course matrix}) ja $\text{\boldmath$\varepsilon$}$ mittauskohinaa. Matriisi \textbf{S} sisältää dipolien amplitudit tiettyinä ajanhetkinä. Sekoitusmatriisi \textbf{A} muodostuu sensorien mittaamista topografioista $\mathbf{A} = [\mathbf{l(p}_1,\text{\boldmath$\eta$}_1),...,\mathbf{l(p}_n,\text{\boldmath$\eta$}_n)]$. Täten $\text{span}(\mathbf{A})$ muodostaa signaaliavaruuden. Oletetaan, että $\text{rank}(\mathbf{A}) = \text{rank}(\mathbf{S}) = n$.

Data-avaruus $\text{span}(\mathbf{Y})$ jaetaan singulaariarvohajotelman avulla kahteen keskenään ortogonaaliseen aliavaruuteen, signaali-avaruuteen $\text{span}(\mathbf{A})$  ja kohina-avaruuteen $\text{span}(\mathbf{A^\bot})$ \citep{Mosher1999SourceMUSIC}. Olkoon matriisin \textbf{Y} singulaariarvohajotelma muotoa $\mathbf{Y = UDV}^T$ ja lähteiden määrä \textit{n}. Matriisin \textbf{D} diagonaalilla olevat \textit{n} ensimmäistä singulaariarvoa kuvaavat signaalia ja loput singulaariarvot kohinaa. Jos kohinaa $\text{\boldmath$\varepsilon$}$ ei ole ja $\mathbf{A}$ oletetaan täysirankkiseksi, vektorit $\mathbf{U}(:,1:n)$ virittävät signaaliavaruuden eli $\text{span}(\mathbf{A}) = \mathbf{U}(:,1:n)$. Loput singulaarivektorit virittävät kohina-avaruuden. \citep{Mosher1999SourceMUSIC, Makela2018TruncatedLocalization} Ortogonaaliprojektio signaaliavaruuteen $\text{span}(\mathbf{A})$ voidaan ilmaista kaavan (\ref{eq:projektio}) avulla:

\begin{equation}
    \mathbf{P}_{sg}=\mathbf{U}(:,1:n)\mathbf{U}(:,1:n)^T
\end{equation} \citep{Makela2018TruncatedLocalization}

Lähteiden määrää ei aina tiedetä, joten se täytyy approksimoida. Todellisissa mittauksissa on aina kohinaa. Datamatriisin singulaariarvoista voidaan päätellä, kuinka monta voimakasta lähdettä löydetään signaaliavaruudesta. Singulaariarvoista voidaan muodostaa pylväsdiagrammi ja lähteiden määrä voidaan approksimoida sen mukaan, missä kohdassa tapahtuu merkittävä pudotus. Toisin sanoen suuret singulaariarvot kuvaavat lähteitä. Valkoisen kohinan tapauksessa pylväiden pituudet putoavat kohdassa, jossa lähteet ovat loppuneet. Värillisellä kohinalla tällaista pudostusta ei nähdä yhtä selkeästi ja täten lähteiden approksimaatio on vaikeampaa.

Olkoon approksimoitujen lähteiden määrä $\Tilde{n}$ ja $\mathbf{Y=UDV}^T$. Projektio signaaliavaruuteen $\mathbf{P}_{sg}$ voidaan nyt muodostaa approksimoitujen lähteiden määrän mukaan $\mathbf{P}_{sg}=\mathbf{U}(:,1:\Tilde{n})\mathbf{U}(:,1:\Tilde{n})^T$, jolloin $\Tilde{n}$ jälkeiset singulaarivektorit virittävät kohina-avaruuden.

Jaetaan tarkastelu kiinteän ja vapaan orientaation MUSIC-algoritmeihin. Tarkastellaan ensin kiinteän orientaation tapausta.

Kiinnitetyn orientaation tapauksessa pisteen $\mathbf{p}$ topografia ei riipu orientaatiosta $\text{\boldmath$\eta$}$. Pisteeseen $\mathbf{p}$ liittyvän topografian $\mathbf{l(p) = L(p)\text{\boldmath$\eta$}}$ kuuluminen signaaliavaruuteen tutkitaan projektio-operaattorin avulla. Pisteen $\mathbf{p}$ topografia $\mathbf{l(p)}$ lasketaan päämallin avulla. Topografia kuuluu signaaliavaruuteen, jos sen projektio signaaliavaruuteen on topografia itse. Toisin sanoen piste $\mathbf{p}$ on lähde, jos $\mathbf{P}_{sg}\mathbf{l(p)} = \mathbf{l(p)}$. Jos $\mathbf{l(p)}$ sijaitsee signaaliavaruudessa, sen normi ei muutu projektiossa eli $||\mathbf{P}_{sg}\mathbf{l(p)}||=||\mathbf{l(p)}||$. Jos $\mathbf{l(p)}$ ei sijaitse signaaliavaruudessa, sen projektion normi on pienempi kuin $\mathbf{l(p)}$:n normi eli
$||\mathbf{P}_{sg}\mathbf{l(p)}||<||\mathbf{l(p)}||$. \citep{Makela2018TruncatedLocalization}

Näistä saadaan muodostettua paikannusfunktio (\textit{localizer function}) \citep{Makela2018TruncatedLocalization}, joka laskee jokaisen pisteen $\mathbf{p}$ kuulumisen signaaliavaruuteen

\begin{equation}
    \mathbf{\mu(p)} = \frac{||\mathbf{P}_{sg}\mathbf{l(p)}||^2}{||\mathbf{l(p)}||^2} 
    \begin{cases}
    =1\text{, jos $\mathbf{p}$ on lähdepiste}\\
    <1\text{, jos $\mathbf{p}$ ei ole lähdepiste}
     \end{cases}
\end{equation}

Vapaan orientaation tapauksessa jokaiselle pisteelle $\mathbf{p}$ määräytyy suuntaus $\mathbf{\eta}$ mittausdatan perusteella. Vektori-MUSIC muodostetaan hyvin samalla tavalla kuin skalaari-MUSIC, mutta sen paikannusfunktio on erilainen ja annetaan kaavalla:

\begin{equation}
    \mathbf{\mu(p)} = \max_{||\eta||=1} \frac{||\mathbf{P}_{sg}\mathbf{L(p)\eta}||^2}{||\mathbf{L(p)\eta}||^2}
    \begin{cases}
    =1\text{, jos $\mathbf{p}$ on lähdepiste}\\
    <1\text{, jos $\mathbf{p}$ ei ole lähdepiste}
     \end{cases}
\end{equation}
Tämä voidaan laskea yleistetyn ominaisarvohajotelmalla Matlabin avulla, joka esitetään liitteessä.

EEG ja MEG ovat ei-invasiivisia aivojen kuvantamismenetelmiä, jotka mittaavat aivojen sähköistä toimintaa. EEG mittaa elektrodien välistä potentiaalieroa päänahalta ja MEG mittaa magneettikenttiä pään ulkopuolelta. Näillä kuvantamismenetelmillä on hyvä temporaalinen resoluutio mutta heikompi spatiaalinen resoluutio. Sähköisen toiminnan lähteiden paikanusta häiritsee näiden kuvantamismenetelmien kokema käänteisongelma. MUSIC on lähteenpaikannusmenetelmä, joka perustuu mittausdatan jakamiseen signaali- ja kohina-avaruuksiin. Algoritmi projisoi potentiaalisen lähdepisteen topografian signaaliavaruuteen ja tarkistaa pisteen sopivuuden paikannusfunktion avulla. Tässä työssä tutustuttiin MUSIC-algoritmiin ja sen rekursiivisiin versioihin, RAP- ja TRAP-MUSIC:iin. Algoritmeja testattiin simuloidulla ja oikealla MEG-mittausdatalla. Simulaatioissa käytettiin pallomallia ja reaalista päämallia.