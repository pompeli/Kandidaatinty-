\section{Lineaarialgebra}
Vektoriavaruuden $\mathbf{V}$ kaikki pisteet voidaan esittää keskenään lineearisesti riippumattomien vektoreiden $\{\mathbf{a}_1,...,\mathbf{a}_n\}$ avulla. Tällöin vektorit $\{\mathbf{a}_1,...,\mathbf{a}_n\}$ muodostavat avaruuden $\mathbf{V}$ kannan.

%aliavaruudet

Olkoon matriisi $\mathbf{A}\in \mathbb{R}^{m\times n}$, jonka rivien määrä on $\mathit{m}$ ja sarakkeiden $\mathit{n}$. 

%käänteismatriisi

Singulaariarvohajotelman avulla matriisi voidaan esittää sen ominaisarvojen ja ortonormaalien matriisien avulla

\begin{equation}
    \mathbf{A = UDV^T,}
\end{equation}
jossa matriisin $\mathbf{A}$ ominaisarvot ovat matriisin $\mathbf{D}$ diagonaalilla. 

Jos matriisi $\mathbf{A}$ on neliömatriisi ja sen determinantti $\det(A)\ne 0$, niin sillä on käänteismatriisi $\mathbf{A^{-1}}$

%pseudoinverse
Pseudokäänteismatriisi $\mathbf{A^{\dagger}}$ on yleistys matriisin käänteismatriisille. Tämä joudutaan muodostamaan silloin, kun matriisi $\mathbf{A}$ ei ole kääntyvä. Pseudokäänteismatriisi saadaan muodostettua singulaariarvohajotelman avulla
\begin{equation}
    \mathbf{A^{\dagger} = VD^{\dagger}U^T}
\end{equation}

%projektio
Projektio signaalialiavaruuteen saadaan kaavalla
\begin{equation}
    \mathbf{\Pi_{sg} = AA^{\dagger}}
\end{equation}

