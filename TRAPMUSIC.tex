\subsection{TRAP-MUSIC}
RAP-dilemman poistamiseksi \cite{Makela2018TruncatedLocalization} kehittivät typistetyn RAP-MUSIC-algoritmin (\textit{truncated RAP-MUSIC, TRAP-MUSIC}), joka poistaa RAP-MUSIC:ssa havaittavan ongelman. RAP-dilemman poistamiseksi vaaditaan muutos RAP-MUSIC:n projektiolausekkeeseen. 

Olkoon lähteiden määrä signaaliavaruudessa yhteensä \textit{n}. Kun lähteitä on löydetty \textit{k} kappaletta, jäljellä olevien lähteiden määrä signaaliavaruudessa on $n-k$. TRAP-MUSIC-algoritmissa projisointi muodostetaan vain jäljellä olevien lähteiden muodostamaan signaaliavaruuteen eli signaaliavaruuden dimensiota typistetään vastaamaan $n-k$ lähteiden virittämää signaaliavaruutta. \citep{Makela2018TruncatedLocalization}.

Iteraatiokierroksen $k-1$ jälkeen signaaliavaruuteen jää $n-k+1$ lähdettä. Täten kierroksella \textit{k} signaaliavaruutta voidaan typistää muuttamalla projektio-operaattoria:

\begin{equation}
    \mathbf{P}_{\text{TRAP}} = \mathbf{U}_k(1:n-k+1)\mathbf{U}_k(1:n-k+1)^T,
\end{equation}
jossa $\mathbf{U}_k$ on kuten yhtälössä \ref{eq:rap}.

Muuten TRAP-MUSIC vastaa täysin RAP-MUSIC-algoritmia. TRAP-MUSIC:n paikannusfunktio kiinteällä orientaatiolla on
\begin{equation}
    \mathbf{\mu_k(p)} = \frac{||\mathbf{P}_{\text{TRAP}}\mathbf{Q}_k\mathbf{l(p)}||^2}{||\mathbf{Q}_k\mathbf{l(p)}||^2}
    \begin{cases}
    =1\text{, jos $\mathbf{p}$ on lähde}\\
    <1\text{, jos $\mathbf{p}$ ei ole lähde}
     \end{cases}
     \label{eq:TRAPfix}
\end{equation}
ja vapaalla orientaatiolla
\begin{equation}
    \mathbf{\mu_k(p)} = \max_{||\eta||=1} \frac{||\mathbf{P}_{\text{TRAP}}\mathbf{Q}_k\mathbf{L(p)\eta}||^2}{||\mathbf{Q}_k\mathbf{L(p)\eta}||^2}
    \begin{cases}
    =1\text{, jos $\mathbf{p}$ on lähde}\\
    <1\text{, jos $\mathbf{p}$ ei ole lähde}
     \end{cases}
     \label{eq:TRAPfree}
\end{equation}