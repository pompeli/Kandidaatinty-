\section{Johdanto}
Aivojen toiminta perustuu sähkökemiallisiin vuorovaikutuksiin hermosolujen eli neuronien kautta. Kynnyksen ylittävä ärsyke saa aikaan hermoimpulssin, joka kulkee hermosolua pitkin ja signaali välittyy neuronilta toiselle synapsin kautta.\cite{} 
Elektroenkefalografian (EEG) avulla saadaan mitattua aivojen potentiaalierojen vaihtelua. EEG:n avulla saadaan mitattua aivojen toimintaa hyvällä aikaresoluutiolla eli sähköinen toiminta voidaan ajallisesti erotella hyvin pienillä aikaväleillä. EEG:tä käytetään usein epilepsian toteamiseen sekä unen analyysiin. \cite{Nunez2006ElectricBrain}

    Neuroneiden tuottamat sähkövirrat saavat aikaan ympärilleen heikkoja magneettikenttiä, jotka voidaan mitata pään ulkopuolelta magnetoenkefalografialla (MEG) \cite{Hamalainen1993MagnetoencephalographytheoryBrain}. MEG-laitteessa käytetään suprajohtavia SQUID-antureita. 
Näillä mittausmenetelmillä saadaan tietoa aivojen aktiivisuudesta, mutta usean signaalilähteen yhtäaikainen toiminta aiheuttaa hankaluuksia paikantaa tietty lähde. 

    Monenlaisia menetelmiä lähteen paikantamiselle on keksitty ja yksi merkittävimmistä on MUSIC-algoritmi (Multiple Signal Classification), joka perustuu mittausdatan aliavaruuksien käyttöön. MUSIC-algoritmilla pystytään mittaamaan useata signaalia yhtäaikaisesti suurella resoluutiolla. \cite{Mosher1999SourceMUSIC}

    Työn tarkoituksena on ottaa selvää, miten MUSIC-algoritmilla saadaan paikannettua signaalin lähteitä EEG- ja MEG-datan analyysissä. Mittausdataa kerätään sekä aika- että taajuusalueella ja analysoidaan MUSIC-algoritmilla MATLAB-ohjelman avulla. Työssä perehdytään myös MUSIC-algoritmin iteratiivisiin menetelmiin, RAP- sekä TRAP-MUSIC -algoritmeihin.
